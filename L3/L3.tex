\documentclass[a4paper]{article}
\title{L3}
\author{Hanwen Jin}
\usepackage[utf8]{inputenc}
\usepackage[T1]{fontenc}
\usepackage{amsmath, amssymb}
\usepackage{tikz}
\usepackage{amsthm}
\usepackage{graphicx}
\usepackage{float}
% figure support
\usepackage{import}
\usepackage{transparent}
\newcommand{\incfig}[1]{%
	\def\svgwidth{\columnwidth}
	\import{./figures/}{#1.pdf_tex}
}
\theoremstyle{definition}
\newtheorem{definition}{Definition}[section]
\newtheorem{example}{Example}[section]
\newtheorem{theorem}{Theorm}[section]
\newtheorem{corollary}{Corollary}[theorem]
\newtheorem{lemma}[theorem]{Lemma}
\newtheorem{exercise}{Exercise}[section]
\newtheorem*{remark}{Remark}
\pdfsuppresswarningpagegroup=1

\begin{document}
	\maketitle
	Tools from calculus:
	Laplace equation: 
	\begin{equation}
		\begin{cases}
			\Delta u=0 \text{ in }\Omega \subset \mathbb{R}^{n}\\
			u\left( x \right) =0 \text{for} x\in \partial\Omega
		\end{cases}
	\end{equation} 
	In general $\Omega$ is an open set, and $\partial \Omega$ is boundary set, it has some regularity. 

	\begin{definition}[]
		$\Omega$ is a $C^{1} $ domain, if $\forall  x\in \partial \Omega$, there exists a system of coordinates $\left( y_1,\ldots,y_{n-1},y_{n} \right) \equiv\left( y',y_{n} \right) $, where$y'$ is the vector contianing all $y$s. with origin at $x$, a ball $B\left( x \right) $ around x and a function $\varphi$ in a neighbourhood $N\subset \mathbb{R}^{n-1}$ of $y'=0'$ such that $\varphi$ is $C^{1} $ in the neighbourhood , $\varphi\left( 0' \right) =0$ and two things happened. 
	\end{definition}
	\begin{enumerate}
		\item $\partial\Omega \cap B\left( x \right) =\left\{ \left( y',y_{n} \right) :y_{n}=\varphi\left( y' \right) ,y'\in \mathcal{N} \right\} $
		\item $\Omega \cap B\left( x \right) =\left\{ y',y_{n}:y_n>\varphi\left( y' \right) y'\in \mathcal{N} \right\} $
	\end{enumerate}
	\begin{remark}
		1 says that locally, $\partial \Omega$ is the graph of the $C^{1}$ function . 2 says that locally $\Omega$ lies on one side of the graph of $\varphi$. 
	\end{remark}
	\begin{remark}
		A $C^{1}$ domain does not have corners, and the tangent line (n=2), the tangent plane (n=3) is always well defined. 
	\end{remark}
\begin{figure}[H]
    \centering
    \incfig{l3f1}
    \caption{L3F1}
    \label{fig:l3f1}
\end{figure}
If $\varphi \in \mathbb{C}^{k} \implies \mathbb{C}^{k} -domain\left( c^{\infty} \text{smooth domain} \right) $

If $\varphi\in \text{Lip}\implies \text{Lipschitz domain}$
\begin{figure}[H]
    \centering
    \incfig{l3f2}
    \caption{L3F2}
    \label{fig:l3f2}
\end{figure}
Integration by parts

$\Omega \in  \mathbb{R}^{n} $ is $\mathbb{C}^{1}$, take a vector fields $F=\left( F_1,\ldots F_{n} \right): \Omega\mapsto \mathbb{R}^{n}, F\in C^{1}\left( \Omega \right) $, we have Gauss divergence theorem. 
\begin{equation}
	\int_\Omega \text{div}F dx=\int_{\partial \Omega}F\cdot \nu d\sigma
\end{equation} 
Where $\nu$ is the outer normal vector. and $d\sigma=\sqrt{1+\left|\nabla \varphi\left( y' \right)  \right| } dy'$ it is the surface measure locally defined. 

The consequences are :
	Take $v\cdot F$ where $v\in C^{1}\left( \Omega \right) $ (scalar). 
	\begin{equation}
		\int_\Omega\text{div}\left( vE \right) =\int_{\partial\Omega} vF\cdot \nu	
	\end{equation} 
\begin{equation}
		\int_{\Omega}^{}\text{div}\left( vF \right)   =\int_\Omega v\text{div} F+\int_\Omega \nabla v \cdot F=\int_{\partial\Omega}vF\cdot \nu
\end{equation} 
		Special case $F=\nabla u$
		\begin{equation}
			\text{div} \nabla u=\Delta u
		\end{equation} 
		\begin{equation}
			\int_\Omega v\Delta u=-\int_\Omega\nabla v\cdot \nabla u +\int_{\partial \Omega}v \partial_\nu u 
		\end{equation} 
		\begin{enumerate}
			\item $\nu=1$: this is the Newmann boundary condition.  
		\begin{equation}
			\int_\Omega\Delta u=\int_{\partial\Omega} \partial_{\nu}u
		\end{equation} 
	\item $v=u$
	 \begin{equation}
		 \int_{\Omega}^{}u \Delta u=-\int_\Omega\left|\nabla u \right|^2  +\int_{\partial\Omega} u\partial_\nu u
		\end{equation} 
		\end{enumerate}
		
Two useful theorems 
\begin{theorem}[ODES]
	Fix $t_0\in \mathbb{R}$, $y_0\in \mathbb{R}^{n}$ $ a,b >0$, and define 
	\begin{equation}
		R=\left\{ \left( t,y \right) :t_0\le t\le t_0+a,\left|y-y_0 \right|\le b  \right\} 
	\end{equation} 
	Consider the ODE
	\begin{equation}
		y'\left( t \right) =f\left( t,y\left( t \right)  \right) ,y\left( t_0\right)=y_0
	\end{equation} 
	Where f is a ctn on R and uniformly Lipschitz in y ($\exists L>0:\left|f\left( t,y_1 \right) -f\left( t,y_2 \right) \right|\le L\left( y_1-y_2 \right)  \forall t_1,y_1,y_2 $ (with maximum equal to $M\ge 0$ in $\mathbb{R}$)
	Then (ODE) has a unique solution $y\left( t \right) $ defined on $\left[ t_0,t_0+T \right] $ where $T=\text{min}\left\{ a,\frac{b}{M} \right\} $
\end{theorem}
\begin{theorem}[Inverse function theorem]
	Let $F:R^{n}\mapsto \mathbb{R}^{n }$ be $C^{1} $ assume $DF\left( a \right)  $ is invertable for some $a\in \mathbb{R}^{n}$, Where $DF$ is a matrix of $[\partial_{x_i}F_j]$. for some $a\in \mathbb{R}^{n}$ let $b=F\left( a \right) $, then 
	\begin{enumerate}
		\item $\exists $ $U,V$ open in $\mathbb{R}^{n}$ such that $a\in U,b\in V$ F is bijecytion on $U$, $F\left( u \right) =V$
		\item If G is the inverse of F in V(it exists by 1) defined by $G\left( F\left( x \right)  \right) =x$ then $G\in C^{1}\left( v \right) $
			Roughly: a $C^{1} $ mapping F is invertable and in a neighbourhood of a point $a\in \mathbb{R}^{n}$ at which the matrix $DF\left( a \right)  $ is invertable. 

	\end{enumerate}
			Consequence: write the equation $Y=F\left( x \right) $ componentwise $Y_i=F_i\left( x_1,\ldots x_n \right) $, the system can be solved for $x_1,\ldots,xn$ in terms of $y_1,\ldots,y_n$ if we restrict $x$ and $y$ to a small neighbourhood of $a$ and $b$ . 

			The solutions are unique and $C^{1}$. 
\end{theorem}
\end{document}

