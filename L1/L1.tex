\documentclass[a4paper]{article}
\title{Lecture 1}
\author{Hanwen Jin}
\usepackage[utf8]{inputenc}
\usepackage[T1]{fontenc}
\usepackage{amsmath, amssymb}
\usepackage{tikz}
\usepackage{amsthm}
\usepackage{graphicx}
\usepackage{float}
% figure support
\usepackage{import}
\usepackage{url}
\usepackage{transparent}
\newcommand{\incfig}[1]{%
	\def\svgwidth{\columnwidth}
	\import{./figures/}{#1.pdf_tex}
}
\theoremstyle{definition}
\newtheorem{definition}{Definition}
\newtheorem{example}{Example}
\newtheorem{theorem}{Theorm}
\newtheorem{corollary}{Corollary}[theorem]
\newtheorem{lemma}{Lemma}
\newtheorem{exercise}{Exercise}[section]
\newtheorem*{remark}{Remark}
\pdfsuppresswarningpagegroup=1

\begin{document}
	\maketitle
	The personal webpage of the lecture is \url{wwwf.imperial.ac.uk/~mcptozel/PDE.thml} His website is \url{m.coti-zelati@imperial.ac.uk}	

	There are four referece in the syalibus. And three coursework. Office hour is wednesday 11-12. 

	\begin{definition}[PDE]
		PDE is an equation of the form 
		\begin{equation}
			f\left( x_1,\ldots x_m,\ldots , u_{x_n}, u_{x_1x_1},  \ldots u_{x_nx_n}\ldots u_{x_1x_2}, \ldots\right) 
		\end{equation} 
		The unknpown is a function $u=u\left( x_1 , \ldots , x_n \right) $ $u_{x_1}, u_{x_2}\ldots$ are partial derivatives. 
	\end{definition}
	\begin{definition}[Order of PDE]
		It is the highest order of differentiation. 
	\end{definition}
	\begin{definition}[Linearity]
		$F$ is linear in $u$ and its partial derivatives. e.g$u_{x_1}+u=x_2^2$. is linear first order . $u_{x_1x_1}+uu_{x_2}=0$ has a quadratic piece, so it is non linear. And it is second order.  
	\end{definition}
	Nonlinear PDEs can also have 
	\begin{itemize}
		\item Semilinear: $F$ is non-linear wrt u, but linear wrt partial derivatives. e.g. $u_{x x}=u^3$. 
		\item Quasilinear: F is linear in the highest differntiation order, but non-linear in other derivatives. e.g. $u_{x x }+\left|u_x \right|^2=u^3 $
		\item Fully nonlinear: F is non-linear in the highest order. $\left|u_{x_1}^2\right| +\left|u_{x_2} \right|^2=x_1^2+x_2^2  $. 
	\end{itemize}
	Linear PDEs/Homogeneous PDES will be biggest part of this course.

	\begin{definition}[Linear operator]
		A linear differntial operator $L$ is a sum of basic derivatives. $L$ acts on functions in a linear way, such that $L\left( u+v \right) =L\left( u \right) +L\left( v \right) \forall u,v$. $L\left( cu \right) =cL\left( u \right) \forall c \in \mathbb{R}$. 
	\end{definition}
	\begin{definition}[Homogenoeous]
		Each term involves $u$ and or its partial derivatives. 
	\end{definition}
	\begin{example}
$u_{x x }+\frac{u}{1+x^2}=0$ is linear and homoge	
	\end{example}
	\begin{example}
	$L=\partial _{x x } \text{for any} u,v\in C^2 \mathbb{R},c\in \mathbb{R}$. 
something is missing here
		$L\left( u+v \right) =\\frac{\partial  }{\partial x} 	$
	\end{example}
	\begin{theorem}[Superposition principle]
		If $u_1, \ldots, n_{n}$ is a solution of $L\left( u \right) =0$, which is a linear homogeneous PDE, then $u=\sum_{n=1}^{k} c_nu_n $ is also a solution. 
	\end{theorem}
	Proof: $L\left( u \right) =L\left( c_1u_1, \ldots,c_ku_k \right) =L\left( c_1u_1, \ldots , c_{k-1}u_{k-1} \right) +L(c_{k}u_k)=0$, repeat this process, you will get $L\left( u \right) =L\left( c_1u_1 \right) + \ldots + L\left( u_nu_n \right) $ which all of the terms are 0. so this is also a solution. 

	The existance of unique solution requires 
	\begin{equation}
		x'=x
	\end{equation} 
	\begin{equation}
		x\left( 0 \right) =x_0
	\end{equation} 
	The top bit may be wrong. 

	\begin{example}
		Heat equation: $u_t=u_{x x},$ $(t,x)\in D\subset R^2$
		Classical solution: $u\in \mathbb{C}^2_{t,x}$ and 
		need to hold forall $t,d \in D$
		\begin{itemize}
			\item 
		$u\left( x,t \right) =t+\frac{1}{2}x^2$ in $\mathbb{R}^2$. 
	\item
		$u_t=1$
	\item 
		$u_x=1$, $u_{x x }=1$. 
	\item
		$\tilde{u}\left( t,x \right) =\exp\left( -\frac{x^2}{4t} \right) / \left( 2\sqrt{\pi t}  \right) \text{ in }\left( 0,\infty \right) \times  \mathbb{R} $. 
		\end{itemize}

		Different behaviour as $\left|x \right|\to \infty $. Also $u+\tilde{u}$ is a solution by superposition principle. 
	\end{example}

	There is no statement like LIpchiz's result, so in this course we will have numberous example. 
\subsection{Boundary condition}%
\label{sub:djflka}
\begin{itemize}
	\item Dirichlet Boundary condition: assign value of $u$ along the boundary. E.g. to find the temperature, assign the temperature in the walls. 
	\item Newmann: assign the normal derivative alone the boundary. ($\nabla =\text{outer normal}$, $\frac{\partial u}{\partial \nu} =\nabla u . \nu$)
	\item Mixed: D on some part of boundary, N on the other. 	
\end{itemize}	
\begin{example}
	For the heat equation again, 
	\begin{itemize}
		\item $u_t=U_{x x }$in $\left\{ t>0 \right\} \times \left( 0,1 \right) $
		\item $u\left( 0,x \right) =f\left( x \right) $
		\item $u\left( t,0 \right) =0 \forall t$
		\item $u\left( t,1 \right) =0 \forall t$. This is an example of a homogeneous boundary condition. 
	\end{itemize}
	We will give a list of PDEs 
	\begin{enumerate}
		\item Transport equation $u_t+v.\nabla u=0$, it is 1st order. $u$ is the concentration in a channel and $v$ is the given velocity of the steam. 
		\item Heat/Diffusion 2nd order:$u_t-D\Delta u=0$. Where $\Delta = \text{Laplacian}=\frac{\partial  }{\partial x_1x_1} +\frac{\partial ^2}{\partial x_2x_2} $, $u$ is the temperature in the medium. 
		\item Wave equation 2nd order , $u_{t t }-c^2\Delta u =0$. $u$ in 1D is the vertical displacement of the string. In 2D it is like the vibration of the drum.  
\begin{figure}[H]
    \centering
    \incfig{l2f1}
    \caption{L2F1}
    \label{fig:l2f1}
\end{figure}
Note this is 2nd order we need the velocity of displacement and also the initial displacement to solve this equation

Also note that the heat equation will become smooth over time, whereas the wave equation will not become smooth.
\item Laplace equarion: $\Delta u=0$
\item Possion equation : $\Delta u=f$ Laplace and Possion equation are the steady state solutuion of wave and heat equation. 
\item Burger's equation: $u_t +uu_x=0$ or $u_{x x }$ this is similar to transport, except the velocity is given by $u$ itself. it is 1st order nonlinear.(shocks). 
\item Navier-Stokes Equations$\frac{\partial }{\partial t } u+\left( u\cdot \nabla  \right) u=-\nabla p+\Delta u$. $\text{div}u=0$. $u=\left( u_1,u_2,u_3 \right) :\left( 0,\infty \right) \times \mathbb{R}^3\to \mathbb{R}^3=\text{velocity of fluid}$, $P=P\left( t,x \right) :\left( 0,\infty \right) \times \mathbb{R}^3\to \mathbb{R}=\text{Pressure}$. 
	\end{enumerate}

	Differential Operators
	\begin{itemize}
		\item 
	$\nabla u=\begin{pmatrix} d_{x_1}u \\ \vdots \\d_{x_n}u \end{pmatrix} $
\item
	$\text{div} \mathbf{F}=\partial_{x_1}F_1+ \ldots +\partial_{x_n}F_n$
\item
	$\Delta u=\partial_{x_1x_1}u+\ldots+\partial_{x_nx_n}$
\item
	for n=3 $\text{curl} \mathbf{F}=\begin{pmatrix} i&j&k\\\partial_x&\partial_y&\partial_z\\F_1&F_2&F_3 \end{pmatrix}$. 
	\end{itemize}
	PDE from modelling 
	\begin{enumerate}
		\item Simple transport: consider fluid flowing at a constant rate $c>0$, in a horizontal pipe in the x-direction. $u\left( t,x \right) =\text{concentration of a substance}$
\begin{figure}[H]
    \centering
    \incfig{l2f2}
    \caption{L2F2}
    \label{fig:l2f2}
\end{figure}
From figure \ref{fig:l2f2}, we can see that 
\begin{equation}
	\int_{b}^{0} u\left( t,x \right) \text{dx}=\int_{b+ch}^{ch} u\left( t+h,x \right) \text{dx}  
\end{equation} 
Differntiate it wrt $b$, 
\begin{equation}
	u\left( t,b \right) =u\left( t+h,b+ch \right) 
\end{equation} 
Take derivative in $h$, 
\begin{equation}
	0=u_t+cu_x
\end{equation} 
	\end{enumerate}
\end{example}
\end{document}

